\documentclass{article}
\usepackage[utf8]{inputenc}
\usepackage{fullpage}
\usepackage[backend=biber,style=apa,backref=true]{biblatex}
\usepackage{listings}
\lstset{language=tex}
\usepackage{url}
\usepackage{graphicx}
\usepackage{amsmath}
\usepackage{hyperref}
\hypersetup{colorlinks=true, citecolor=blue}

%for including wordcount in file
\immediate\write18{texcount -template="{word} words in main body, excluding headers and bibliography." \jobname.tex -out=\jobname.sum}
\usepackage{verbatim}
\newcommand\wordcount{\verbatiminput{\jobname.sum}}


\author{George Tyler}
\date{\today}

\addbibresource{diss.bib}
\begin{document}

\title{Social Media, Public Sentiment, and Social Distancing: \\ Evidence from 10 Million Tweets}

\maketitle
%This is a writing sample to be submitted to Economics Masters programmes. The writing sample will be assessed for a comprehensive understanding of the subject area; understanding of problems in the area; ability to construct and defend an argument; powers of analysis; and powers of expression. I should display the word count on the document. 
\abstract{Does social media predict risk-taking behaviour? I investigate this question in the context of COVID-19 by exploiting a large panel of tweets. Using inferred and explicit geolocation data embedded in the tweets, I study the extent to which public expressions of sentiment such fear, anger, and optimism influence social distancing, as measured by GPS-located smartphone data. In this 2000-word excerpt, I motivate the research design, conduct a literature review, and present descriptive statistics on a portion of the dataset.}

\section{Introduction}
\subsection{Overview}
The early stages of the COVID-19 pandemic saw an unprecedented shift in behaviour for most citizens of the United States. In a short period of time, a large number changed their habits of working, socialising, and travelling. They did so both as a result of government restrictions in the form of non-pharmaceutical interventions (NPIs) and as a private response to the spread of the pandemic. Economists have taken interest in how citizens formed these behaviour changes, and the role that beliefs and risk attitudes played in determining the response to public policy. A new way to measure belief formation and public sentiment is with social media, an increasingly common platform for expression of opinion. It is plausible that those who express less risk-averse sentiment towards COVID online will be inclined to respond in a more lax fashion to social distancing and other public health regulation. In this dissertation, I investigate the role of online expressions of risk attitude on public behaviour in the early months of the pandemic. Specifically, I study whether a measure of risk-seeking sentiment on Twitter is linked to decreased social distancing behaviour, agreggated at the county/week level. 

A key vector for expressing sentiment is social media, with Twitter and Facebook's suite of products\footnote{Facebook, Facebook Messenger, Instagram, and WhatsApp} being the most widely-adopted, each platform having over 80 million monthly active users in the US. A survey by the Pew Research Foundation indicates that 22\% of US adults use Twitter, with 42\% of these using it on a daily basis \parencite{perrinShareAdultsUsing2019}. On Twitter, users can share their own text, with the option to link to a website; alternatively, they can `retweet' another user's text or link. Users can also use `hashtags' in their tweet, which connects their tweet to a particular topic. If the user has allowed it, Twitter also records the location of the tweet; and it is also possible for the user to set their location on their profile. In this way, it is possible to create a panel of geographically-located tweets about a particular topic. 

I exploit GeoCov19, a dataset of 524 million geolocated tweets, to measure the local public sentiment on COVID in the US. The tweets cover the period from 1st February to 1st May, the period I focus on. The particular subset of the data I use contains X tweets in total; X are exactly geolocated (the user has provided a GPS location), and X are inferred from the location tab in the user's profile. The tweets were collected using the Twitter API, querying for a random sample of 1\% of the tweets containing any of a list of 800 COVID-related keywords. I also use anonymous smartphone location data, collected by the company SafeGraph, as a measure of the extent of social distancing in an area. I present two measures of social distancing at county level: first, the median minutes spent at home during 8am-6pm; second, the proportion of measured devices that stayed at home all day \parencite{safegraphinc.SocialDistancingMetrics2020}.Demographic controls are also acquired and presented from the American Community Survey and the 2010 US census. 

I use dictionary-based text analysis to assess the level of risk sentiment in a tweet. More sophisticated methods of text analysis like latent factor modelling are discussed in the Methods section. In the absence of a lexicon of risk preference, the NRC Emotion Lexicon \parencite{mohammadCrowdsourcingWordEmotion2013} is used. This is a widely-used mapping of English words to eight basic emotions (anger, fear, anticipation, trust, surprise, sadness, joy, and disgust). Starting from a set of tweets that mention COVID, I assign tweets containing fear-associated words to a risk-averse sentiment. The base unit of analysis is the county-week; as such, I measure the proportion of tweets that contain fearful language in each county and week. 

It is plausible that social media is a valid measure for risk appetite. The intuition is that the textual content of a social media post broadly reflects the poster's current opinion of a topic: for example, in response to the first confirmed US COVID death on February 26th, a user may express fearful, or uncertain, sentiment: for example, `i think i have 70 panic attacks every day because of how scared i am for my mom to catch corona', a Tweet in the dataset -- or a neutral sentiment. This opinion of the topic, particularly their level of fear, maps to a user's broader expectations about the course of the pandemic: while other emotions like joy, anticipation, and trust may rely on the context of the discussion, expressions of fear are plausibly consistent in mapping to risk-averse sentiment. When restrictions are implemented, users who initially formed pessimistic expectations may be more inclined to adhere more to them than a user who formed optimistic or netural expectations. 

The primary econometric specification is a panel model with county and week fixed effects;
\[Y_{it} = \alpha + \beta r_{it} + \mu c_{it} + \tau_i + \delta_t +  X_{it}\gamma + \epsilon_{it}\]
where \(Y_{it}\) is a vector of social distancing metrics, \(\beta r_{it}\) is the risk perception measure, (i.e. the proportion of total tweets containing fearful language), \(\mu c_{it}\) the number of COVID cases, \(\tau_i + \delta_t\) county and week-level fixed effects, and \(X_{it}\gamma\) demographic controls.

This research contributes to the recent economics literature seeking to explain the disparities in social distancing in the early stages of the pandemic in the US. In particular, partisanship has been shown to be a significant factor on the practice of social distancing: \textcite{allcottPolarizationPublicHealth2020,barriosRiskPerceptionLens2020,painterPoliticalBeliefsAffect2020} show that areas with more Republicans engaged in less social distancing, are associated with lower perceptions of risk of the pandemic, and exhibited less remote transactions. \textcite{simonovPersuasiveEffectFox2020,ananyevSafestTimeFly2020} also measure the causal effect of the right-wing Fox News network on social distancing during the pandemic. This paper builds on \textcite{barriosRiskPerceptionLens2020} in particular, which shows that online risk perception is predicted by Trump voter share: by measuring risk perception with a high-frequency geolocated dataset, my approach controls for political alignment and assesses the effect of risk perceptions on their own. In essence, the above papers argue that political beliefs affect complicance with Social Distancing orders; I measure expressions of sentiment regarding COVID risk, and given this data I ask two questions: first, does local risk sentiment predict social distancing behaviour beyond political affilitation; second, do differing interpretations of political messages -- like Trump's messages in March and April downplaying the virus -- colour local risk sentiment?

%This paper is particularly close to two research papers, This dissertation takes a similar approach, but uses Twitter as a news source instead. More broadly, this dissertation contributes to the literature on the effects of misinformation and political affiliation on public behaviour.

This dissertation also contributes to the rapidly-expanding field of text analysis in economics, and presents an example of how the rich sentiment data encoded in social media communication can inform insights into public behaviour. This topic is particularly mature in finance -- where sentiment data from public company documents, news media, and social media have been shown to predict stock market reactions \parencite{bollenTwitterMoodPredicts2011} -- and monetary economics, where central bank statements, coded according to their dovishness or hawkishness, predict fluctuations in Treasury securities \parencite{luccaMeasuringCentralBank2009,gentzkowTextData2019}. On the topic of empirical economics, this paper takes a similar approach -- by using online data to predict local sentiment -- as \textcite{stephens-davidowitzCostRacialAnimus2014}, which uses Google search data to proxy an area's racial animus, and uses this to estimate the Obama vote share. I use geolocated Twitter sentiment to proxy the local attitude to COVID in a given week, and test to see if this predicts social distancing practice. 

The argument of the dissertation rests on the following assumptions: first, that social media data is a valid proxy for local risk appetite, and that fear-associated language in COVID-related tweets is an effective estimator of the risk appetite encoded in the tweet. It is also important to note a possible selection effect in the dataset: tweets about COVID may attract a greater level of fear-related language and not reflect an individual's true opinion about social distancing and other preventative measures. I address these assumptions and drawbacks and discuss methods to alleviate them in the Results section.
% Misinformation: what forms did it take in the early pandemic? How do I characterise it? How do I filter for it?

\section{Literature Review}
\subsection{Misinformation and media bias}
\subsection{Digital trace datasets}
\subsection{COVID-19 policies and social distancing}

\section{Data}

% The early stages of the COVID-19 pandemic saw a rapid and significant shift in economic incentives for those in countries where the virus had spread. The quality of information on the pandemic was key to determining the early response to the various non-pharmaceutical interventions (NPIs) put in place; I study the role of varying levels of information quality circulated on social media. Over the last decade, consumption of news has shifted from traditional sources of media to the online sphere, and, accordingly, the market for news has become more fragmented. A degree of polarisation on both ends of the political spectrum has occurred, and with it a certain level of online misinformation. The typical consumption pattern for misinformation is through social media, particularly Twitter and Facebook (in the English-speaking world). Using inferred and explicit geolocation data, I study the extent to which greater misinformation spread influences compliance with NPIs via the SafeGraph metric of social distancing.

% %The COVID-19 pandemic has had an enormous impact on the economic lives of everyone in the US, both from health costs and widespread stay-at-home orders. Business closures and social distancing mandates have also substantially changed behaviour when not locked down. Until a suitable vaccine is developed, non-pharmaceutical interventions (NPIs) will remain a substantial policy problem. There are a vast number of pertinent questions that accompany implementing NPIs. At what stage of infection should a government restrict economic activity? How effective are intermediate measures, and at what point should they be converted to a full lockdown? What causes effective adherence to intermediate interventions? How much do they cost, and does this cost vary across regions? If so, what determines this difference?



% In this dissertation, I study multiple aspects of this policy problem. First, I consider the impact of weather on the effectiveness of intermediate measures. Weather affects transmission directly by affecting the half-life on surfaces and airborne transmission, but also indirectly, as a proxy for socialising outdoors and other similar activities. This works on a day-to-day channel and as a historical characteristic. Historical temperatures point to norms of summer activities: areas with very hot summers may socialise more indoors than temperate areas. Conversely, daily temperature and humidity impact indoors vs. outdoors mobility and physical transmission itself. The interaction of good weather with intermediate NPI measures presents a tradeoff, particularly to younger age groups. 

% Using a difference-in-differences framework, I test to see if the variation in social distancing is explained by daily weather conditions. I also test to see if historical weather conditions, alongside commuting norms and neighbourhood norms, explain variation in mobility patterns. Off the back of these results, I include a weather metric in an epidemiological model.

% Second, I examine the role of regional factors on NPI effectiveness. Using spatial techniques, I study the impact of population density, housing mode, and job position on effectiveness of NPIs. I aim at a first step to replicate the results of \cite{almagroRacialDisparitiesFrontline2020}\footnote{``COVID disparities stem from patterns of commuting and housing crowding''}. The authors establish an initial effect from commuting to essential work, and after a lockdown, an effect from crowding in housing. I aim to build on this analysis by controlling for commuting (with SafeGraph data) and examining a causal effect of housing modality.\textbf{\textit{Next Step:} Think about and set out a research design for this experiment. I'm thinking something along the lines of studying people that move house?}
%  %I exploit the variation in timing and duration of stay-at-home orders to infer a causal effect of lockdown.
 
% Third, I use the Imperial College COVID-19 model to estimate latent infection levels, and to simulate counterfactual deaths. I trace the economic cost against the benefit of counterfactual lives saved and infections avoided, and test the robustness of my results against alternative value of statistical life (VSL) specifications. I address \textcite{pindyckCOVID19WelfareEffects2020}'s criticism of using VSLs to measure health policy costs.
% \section{Data}

\section{Methods}
% I use a hierarchical Bayesian epidemiological model to estimate latent infections. Broadly, the economics literature has preferred to use variants of the Susceptible-Infected-Recovered (SIR) model to estimate epidemiological parameters with Maximum Likelihood, rather than take a `structural equations' Bayesian approach. A common criticism of the Bayesian approach is that it fails to take behavioural concerns into account. In the context of a benefit-cost analysis, I now show that it is possible to apply behavioural tradeoffs to the model while retaining the benefits of the hierarchical model -- namely the . 

\section{Results}

\section{Discussion}

\wordcount
\printbibliography
\end{document}